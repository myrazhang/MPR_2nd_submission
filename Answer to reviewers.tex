\documentclass[suppldata]{interact}

\usepackage{epstopdf}% To incorporate .eps illustrations using PDFLaTeX, etc.

\usepackage[natbibapa,nodoi]{apacite}
\setlength\bibhang{12pt}
\renewcommand\bibliographytypesize{\fontsize{10}{12}\selectfont}

\usepackage{color}

\begin{document}
\noindent
Dear DE, AE and reviewers,

Thanks to your valuable %review 
comments, this submission has undergone a major revision. In this %new 
revised version, the MP models are developed based on general timed Petri nets (TPN) instead of the proposed representation of discrete event dynamic systems (DEVS) that was used in the previous submission. This change reduces the difficulty of applying the proposed approach and can be better understood by people familiar with DEVS.


The concerns about application, usefulness, benefit and computational complexity were mentioned by the editors and reviewers for several times. To answer such concerns, we have seen many applications of the use of mathematical programming (MP) for the representation of system dynamics in literature. Most of the works couples the system dynamics and optimization, as in \cite{di2020makespan}, \cite{weiss2015buffer}, \cite{bemporad1999control}, \cite{alfieri2020time}. All those works solve problems in industrial applications, thus, the applicability has been proved. The properties of DEVS can also be analyzed through MPs, as in \cite{basile2012k}.


The usefulness %and benefit 
of this work can be justified in two aspects, namely generalization and convenience. TPN is a widely--used general modeling diagram of DEVS, and the proposed work would be easy to apply for people in this field.


The computational complexity of using MP to optimize DEVS has been %noticed 
addressed in many works, and they dealt the issue with classic MP techniques such as Benders decomposition, linear approximation, columns-row generation. We cannot deny that the computation issue really exists, but the ways to overcome are also vast. Furthermore, the translation is bidirectional. One can translate simulation into MP to optimize a system, and also translate the resulting MP into simulation. The value of translating MP into simulation is that the research on MP (such as duality and sensitivity) can %dig
increase the value of simulation, which is currently usually regarded only as a performance evaluation tool. 


In the following, we answer the comments in details. 



\section{DE comments}
\textit{The merit of the proposed technique should be clarified either in a general setting or on specific classes of systems like Petri nets or finite automata, and comparisons with extant tools and algorithms should be made.}

\noindent
\textbf{Answer:} %In this submission, 
In the revised manuscript our approach %works
has been reformulated to have it working under general settings deterministic timed Petri net. With an example of the $G/G/m$ queue, the differences between the proposed model and the state-of-the-art model are discussed.


\section{AE Comments}
\textit{The contribution and usefulness of the proposed approach is not convincing and should be improved. }

\textit{1. The authors should convincingly demonstrate the benefit of the proposed MPR over other formalism like the automata and PN while also considering the complexity burden of solving the resulting mathematical program. }

\noindent
\textbf{Answer:} %In this submission, 
In the revised manuscript, we justify the benefit and computational burden of the proposed approach as follows: 

The benefits of using MP model to represent discrete event systems are many. As already mentioned above, when coupled with optimization, the MP-based algorithms can be faster and %gain 
reach better solution than black--box optimization algorithms. Furthermore, black--box approaches have limited capability is solving constrained optimization problems, %but 
while MP-based approaches can easily deal with them (\cite{zhang2020models}). Third, the vast theoretical and methodological results developed in the MP field can be introduced into the study of DEVS through simulation. For instance, using sensitivity analysis of MP models, DES can %be
become a gradient estimation tool, as suggested in \cite{chan2008optimization}. In fact, we do not proposed to totally replace simulation with MP, but to enrich the toolbox for analyzing and optimizing DEVS. Finally, the application of MP models of DEVS is not limited to optimization. \cite{basile2012k} proposed the sufficient and necessary condition of K-diagnosability of TPN based on MP. 


The major concern about the application of MP is the computational complexity, %and the solving procedure 
as the solution procedure can be time-consuming or even unbearable. However, many approaches from optimization community are available to improve the efficiency. Linear programming approximation (\cite{alfieri2012mathematical}), Benders decomposition (\cite{weiss2015buffer}), row-column generation (\cite{alfieri2020time}), have been studied to solve optimization problems in %reality
real contexts based-on MP model of DEVS.

~\\

\textit{2. Furthermore, the discussion about the application of the approach should be significantly strengthened. Currently, this is rather vague and there is no comparisons with existing approaches that are used to address these problems.}

\noindent
\textbf{Answer: With an example of the $G/G/m$ queue, the differences between the proposed model and the state-of-the-art model are discussed.}


\section{Reviewer 1}
\textit{However, I think that the authors should clarify better their contribution with respect to the standard translation of other modelling formalisms into mathematical programming (see references below). From a certain point of view, mathematical programming is also a modelling formalism, even it is usually considered more a computational tool, anyway it is evident that it cannot replace finite state automata or Petri nets to model DES. Hence, my question is, there is a way to translate, as for example a PN using the proposed technique? If yes, is it more efficient? If not, what is the effective application domain of the proposed technique, in addition to queueing systems?}


\textit{References: \cite{basile2012k}, \cite{basile2012k}, \cite{bemporad1999control}}

\noindent
\textbf{Answer:} As the reviewer suggests, %this submission proposed 
in the revised manuscript we propose the translation of TPN into MP. 


The difference between the proposed approach and the suggested references are %as follows.
detailed in the following. \cite{bemporad1999control} proposed a mixed integer programming modeling framework for hybrid systems, whose states are mixed integer, in discrete time, while this work deals with continuous--time discrete--state systems. An MP model was proposed to minimize the makespan of single-server manufacturing systems in \cite{di2020makespan}, %but this
while our work is more general and can model multi-server systems. \cite{basile2012k} proposed the model based on PN, not TPN, because %it deals with 
the paper is about diagnosability %other 
rather than time-dependent performance measures, such as waiting time or throughput. 


The effective application domain of the proposed technique is basically related to optimization. The MP model of TPN %is easy to expand 
can be easily expanded to an MP model of TPN optimization. Based on the proposed MP model, an example with the extended MP model to optimize the $G/G/m$ queue is presented in Section 4.3.


\section{Reviewer 2}

\textit{1. The main criticisms I can to formulate about this MPR approach is that
(i) it produces only fixed-length simulations of a system, and (ii) it
results in high-dimension linear programs (O(NxE) where N is the
length of the execution and E the number of event types). This puts
stringent restrictions on the possible applications of this
encoding. For instance, steady-state behavior can only be
approximated, at the expense of very large MPR systems. This casts a
serious doubt about the usefulness of the whole approach, in
comparison with a DES simulator, implementing an operational
semantics of the particular DES considered in the paper.}

\noindent
\textbf{Answer:} The reviewer's comment is reasonable. We %justify 
discuss the computational burden as follows in the submission:

The major concern about the application of MP is the computational complexity, %and the solving procedure 
as the solution procedure can be time-consuming or even unbearable. However, many approaches from optimization community are available to improve the efficiency. Linear programming approximation (\cite{alfieri2012mathematical}), Benders decomposition (\cite{weiss2015buffer}), row-column generation (\cite{alfieri2020time}), have been studied to solve optimization problems in %reality
real contexts based-on MP model of DEVS.

~\\

\textit{\textcolor{blue}{For this reason, I recommend the rejection of the paper, and encourage the authors to replace part of the queueing network case-studies by an in-depth technical discussion of the possible applications of this MPR encoding.}}

\noindent
\textbf{Answer:}

~\\

\textit{2. Detailed Comments}


\textit{The paper reads clearly. There are only very few typos (see at the end).}
\textbf{Answer: we have correct the typos.}



\textit{From a technical point of view, there is one thing that needs to be clarified, page 8: How do you choose constant M? Understandably, it should be large enough. However, I would expect to find in the paper a discussion about a lower bound on M.}

\noindent
\textbf{Answer: In the subsections of Section 3, the value of M is discussed each time it appears.}


\textbf{Section 3.1: The value of M in constraints (A1) and (A2) can be set to a value that is larger than the simulation time span.}

\textbf{Section 3.2: The value of M can be set to a value that is larger than the simulation time span for constraints (B1) and (B2).  The value of $M$ can be chosen as the upper bound of marking of place $p$ minus $(W^{p,t}-1)$ for constraints (B4). }


\textbf{Section 3.3: The value of $M$ in constraints (C3) and (C4) can be set to $K$. }

~\\


\textit{The queueing networks case-studies are carefully detailed. Perhaps
some space could be gained here, and used Section 4, to allow a
detailed and technical discussion about the applications of the MPR
encoding.}

\noindent
\textbf{Answer:}


~\\
\textit{Here is a list of typos: p 5, Section 2.3, line 2: ...proposed. This...}

\noindent
\textbf{Answer:}



%Reference
\bibliographystyle{apacite}
\bibliography{RAP, Biblio_Loop}

\end{document}